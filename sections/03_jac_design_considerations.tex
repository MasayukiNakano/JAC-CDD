\section{JAC Design Considerations}

The JAC must provide sufficient suppression of the first-order spatial modes that carry beam jitter, while remaining compatible with the mechanical, optical, and operational constraints of HAM1. In this section, we evaluate three candidate cavity geometries---a triangular cavity, a optimized bow-tie cavity, and a direct replica of the aLIGO PMC---and compare their performance in terms of jitter attenuation, sensitivity to polarization effects, and practical implementability. 

We begin by describing the mathematical formalism for beam jitter and the coupling of mirror motion into higher-order spatial modes. Using this model, the output jitter of each cavity design is estimated, including both the filtered residual jitter from the PSL and the jitter introduced by motion of the JAC optics themselves. We then analyze polarization-induced detuning, which differs significantly between triangular and bow-tie geometries, and assess the resulting contribution to relative intensity noise. Finally, we summarize the optical parameters of each configuration and discuss their expected performance.

As shown in the following subsections, although the triangular cavity provides an intrinsic advantage in suppressing polarization-induced detuning, all three candidate designs meet the jitter attenuation requirement.

 {\bf The overall performance, combined with considerations of risk, availability, and ease of implementation, ultimately favors a JAC design identical to the current aLIGO PMC.}
\subsection{Jitter Attenuation}

\subsubsection{Jitter Expression}
Before discussing the JAC geometry, we describe the mathematical representation of beam jitter. We assume that the laser propagates along the $z$-axis in the fundamental transverse electromagnetic mode (TEM\(_{00}\)). For simplicity, only the $xz$-plane is considered. A finite beam displacement $\Delta x$ and angular deviation $\Delta\theta$ around the beam waist lead to coupling from the TEM\(_{00}\) mode into the TEM\(_{10}\) mode~\cite{misalignment}:

\begin{equation}
    E' = E_0\left[\left(1 - \left(\frac{\Delta x}{w_0}\right)^2 - \left(\frac{\Delta \theta}{\alpha}\right)^2\right)e_0
    + \left(\frac{\Delta x}{w_0} + i\frac{\Delta\theta}{\alpha}\right)e_1\right],
\end{equation}

where $E'$ is the electric field of the perturbed beam, $E_0$ is the input field amplitude, $w_0$ and $\alpha$ are the beam waist size and divergence angle, and $e_0$ and $e_1$ are the spatial profiles of the TEM\(_{00}\) and TEM\(_{10}\) modes, respectively. Jitter in the $y$-direction is described analogously using the TEM\(_{01}\) mode. Thus, beam jitter can be equivalently expressed as fluctuations in the amplitudes of the TEM\(_{10}\) and TEM\(_{01}\) modes.

Because the JAC acts as a spatial mode filter, suppressing higher-order modes, jitter attenuation is quantified by the suppression factor of the TEM\(_{10/01}\) modes.

For the following discussion, we describe the first-order mode content generated by mirror motion at a distance $z$ from the beam waist. A rotation of the mirror by $\Delta\theta$ shifts the beam waist by $z\Delta\theta$ and introduces a tilt of $\Delta\theta$ around the waist. The resulting TEM\(_{10}\) amplitude is

\begin{equation}
\begin{aligned}
A_1 &= \sqrt{\left(\frac{z\Delta\theta}{w_0}\right)^2 + \left(\frac{\Delta\theta}{\alpha}\right)^2}
     = \frac{k w_0}{\cos\eta(z)}\,\Delta\theta,
\end{aligned}
\label{eq:mirrormotion}
\end{equation}

where $k$ is the wavenumber and $\eta(z)$ is the Gouy phase at position $z$.

\subsubsection{Output Jitter Estimation}\label{sec:output_jitter}
In this subsection, we outline the method used to estimate the jitter noise at the JAC output. The output jitter consists of two contributions:  
(1) residual jitter after spatial filtering of the incident TEM\(_{10/01}\) modes by the JAC, and  
(2) jitter introduced by motion of the JAC itself.

We begin with the residual jitter. During O4, the dominant source of incident jitter was identified as motion of the upper mirror of the PSL periscope, across the entire observation band. The periscope mirror motion is reconstructed using the IMC WFS signals as witnesses and calibrated via PZT injection measurements~\cite{alog:IMCproj}. Once the JAC geometry is fixed, the beam waist size and Gouy phase at the periscope location are determined. Assuming perfect mode matching between the PSL beam and the JAC eigenmode, the first-order mode content at the JAC input can be computed using Eq.~\eqref{eq:mirrormotion}. Combined with the TEM\(_{10/01}\) attenuation gain determined by the JAC geometry and finesse, the residual jitter at the JAC output can be estimated.

Next, we consider jitter introduced by motion of the JAC optics. The JAC is assumed to be mounted on the ISI table newly installed in HAM1. (Seismic isolation requirements and ISI performance are discussed in the next section.) The jitter contribution from JAC motion is computed from the ISI motion spectrum and the cavity parameters of the JAC. More detailed discussion of seismic isolation is provided in Sec.~\ref{sec:seismic_isolation}.
\subsection{Polarization Separation}
\label{sec:pol}

Odd-mirror cavities such as the triangular design exhibit polarization-dependent resonance frequencies, whereas even-mirror cavities such as the bow-tie design have resonances for the two linear polarizations that lie within a single cavity linewidth. When a beam with imperfect polarization is injected, this multi-resonant condition introduces an offset in the locking error signal and detunes the cavity. A detuned cavity converts residual laser--cavity frequency error into intensity fluctuations. From this viewpoint, the triangular cavity has an intrinsic advantage. In this subsection, we quantify this effect for comparison.

Assuming the JAC is critically coupled and operated close to resonance, the transmitted power for a purely p-polarized beam is
\begin{equation}
    P^p_\mathrm{T}(\xi) = \frac{P}{1+\xi^2} \equiv P\, L(\xi),
\end{equation}
where \(L(\xi)\) is a normalized Lorentzian, \(P\) is the input laser power, and \(\xi\) is the detuning normalized by the cavity linewidth. The cavity half-linewidth \(\omega_c\) is related to the free spectral range (FSR) \(\omega_\text{FSR}\) and finesse \(\mathcal{F}\) by \(\omega_c = \omega_{\mathrm{FSR}}/(2\mathcal{F})\).

If the resonance frequencies of two orthogonal polarizations differ by \(\Delta\xi\), the transmitted power of the s-polarized beam is
\begin{equation}
\begin{aligned}\label{eq:11_20_17_18_40}
     P^s_\mathrm{T}(\xi)
     = \frac{P}{1+(\xi-\Delta\xi)^2}
     \equiv P\, L(\xi-\Delta \xi)
     \approx P \left[ L(\xi) - \frac{dL}{d\xi}(\xi)\, \Delta\xi \right],
\end{aligned}
\end{equation}
where the last expression assumes \(|\Delta\xi|\ll 1\).

If the input beam has mixed polarization, with a rotation angle \(\theta\) from p-polarization, then the transmitted power becomes
\begin{equation}
\begin{aligned}
    P_\mathrm{T}(\xi)
    &= \cos^2\theta\, P^p_\mathrm{T}(\xi)
     + \sin^2\theta\, P^s_\mathrm{T}(\xi) \\
    &\simeq P \left[ L(\xi) - \frac{dL}{d\xi}(\xi)\, \Delta\xi \sin^2\theta \right] \\
    &\simeq P\, L\!\left(\xi - \Delta\xi \sin^2\theta\right).
\end{aligned}    
\end{equation}
Thus, the effective resonance peak shifts by \(\Delta\xi \sin^2\theta\). In practice, the cavity is locked to the maximum of the transmitted power with the Pound-Drever-Hall locking, so the operating point is detuned by this amount relative to the p-polarization resonance for a mixed-polarization input.

Under this locking condition, the transmitted electric field can be written as
\begin{equation}
\begin{aligned}\label{eq:11_20_17_40_02}
    \vec{E}_\text{out}
    = \sqrt{P^p_\mathrm{T}(\Delta\xi \sin^2\theta)}\, \vec E^p\cos\theta
    + \sqrt{P^s_\mathrm{T}(\Delta\xi \cos^2\theta)}\, \vec E^s\sin\theta,
\end{aligned}
\end{equation}
where \(\vec E^p\) and \(\vec E^s\) are the unit polarization vectors for p and s, respectively. The output beam from the JAC is then injected into the Input Faraday Isolator (IFI). The IFI (and the IMC) preferentially transmits a single linear polarization and thus filters out the orthogonal component\footnote{The p-polarization component at the JAC output is rotated into s-polarization by the HAM1 output periscope before injection into the IMC. Consequently, at the IMC and IFI the roles of ``p'' and ``s'' are effectively interchanged, but the analysis in terms of two orthogonal linear polarizations remains valid.}.

The JAC length control will also have a residual error \(\delta \xi(t)\) in normalized detuning. Therefore, the power incident on the interferometer can be written as
\begin{equation}
\begin{aligned}\label{eq:11_20_17_53_08}
    P_\text{IFO}(t)
    &= P^p_\mathrm{T}\bigl(\delta \xi(t) + \Delta\xi \sin^2\theta\bigr) \\
    &\simeq P \left[ L(\Delta\xi \sin^2\theta)
      + \frac{dL}{d\xi}(\Delta\xi \sin^2\theta)\, \delta \xi(t) \right],
\end{aligned}
\end{equation}
where the second term in brackets represents the intensity noise coupled from the residual cavity length fluctuations.

The derivative of the Lorentzian evaluated at the detuned operating point is
\begin{equation}
\begin{aligned}\label{eq:11_20_18_18_16}
    \frac{dL}{d\xi}\bigl(\Delta\xi \sin^2\theta\bigr)
    = \frac{-2\Delta\xi \sin^2\theta}{\left[1 + (\Delta\xi \sin^2\theta)^2\right]^2}
    \approx -2 \Delta\xi\, \theta^2,
\end{aligned}
\end{equation}
where we used \(\sin\theta \simeq \theta\) and \(|\Delta\xi\, \theta^2|\ll 1\). This derivative gives the transfer function from residual JAC length noise to relative intensity noise (RIN) at the interferometer input.

The normalized detuning is related to the cavity length fluctuation \(\delta l\) by
\begin{equation}
\begin{aligned}\label{eq:11_20_18_22_36}
    \delta\xi
    = \frac{\delta l}{L_\text{linewidth}}
    = \frac{\delta l}{\lambda/(4\mathcal{F})}
    = \frac{4\mathcal{F}}{\lambda}\, \delta l,
\end{aligned}
\end{equation}
where \(L_\text{linewidth} = \lambda/(4\mathcal{F})\) is the cavity linewidth in the unit of length.

Combining Eqs.~\eqref{eq:11_20_18_18_16} and \eqref{eq:11_20_18_22_36}, the transfer function from cavity length fluctuations to RIN is
\begin{equation}
\label{eq:dpdl}
\begin{aligned}
    \left.\frac{1}{P}\frac{dP}{dl}\right|_{\theta}
    &= \frac{1}{P}\frac{dP}{d\xi}(\theta)\, \frac{\delta\xi}{\delta l}
     \simeq 2\Delta\xi\, \theta^2 \times \frac{4\mathcal{F}}{\lambda} \\
    &= 2.9 \times 10^{-6}\,\mathrm{/nm}
       \left(\frac{\Delta\xi}{0.01}\right)
       \left(\frac{\mathcal{F}}{125}\right)
       \left(\frac{1064\,\mathrm{nm}}{\lambda}\right)
       \left(\frac{\theta}{1\,\mathrm{deg}}\right)^2 .
\end{aligned}
\end{equation}

This analytic model was validated using \textsc{Finesse}~\cite{finesse}. As shown in Fig.~\ref{fig:polTF}, both approaches give consistent transfer functions.
\begin{figure}
    \centering
    \includegraphics[width=0.5\linewidth]{Figs/finessevsmodel.pdf}
    \caption{Transfer function from cavity length fluctuations to output relative intensity noise obtained from Eq.~\eqref{eq:dpdl} and from \textsc{Finesse} modeling.}
    \label{fig:polTF}
\end{figure}
\subsection{JAC Design Candidates}

Based on the discussion above, the output beam jitter and the relative intensity noise caused by the mispolarization were estimated for three different JAC configurations. The key parameters of each design candidate are summarized in Table~\ref{tab:JACdesign}.

Two classes of cavity geometries were considered: triangular and bow-tie. The overall cavity size is chosen to be comparable to the aLIGO PMC. Since spare PMC units are available, a direct replication of the current PMC design is considered as one candidate. In addition, because the PMC mirrors can be replaced without modifying the mechanical body, a bow-tie configuration using mirrors with different radii of curvature (RoC) was also evaluated. The RoC values are selected to optimize the Gouy phase and overlap with higher-order modes.

For the triangular cavity, dimensions similar to the PMC are chosen, with RoC values selected in the same manner as for the bow-tie design. The finesse is chosen such that the power density on the mirrors remains below 1~MW/cm\(^2\) for 100~W of incident power. More detailed design considerations are discussed in~\cite{G2301529}.

\subsubsection{Jitter Attenuation}

The expected jitter attenuation performance is shown in Fig.~\ref{fig:jitter_att}. The newly designed cavities exhibit larger waist sizes than the PMC, and their higher achievable finesse yields stronger attenuation of TEM\(_{10/01}\) modes. Between the two new designs, the bow-tie geometry places the mirrors farther from the beam waist, resulting in a larger beam size on the mirrors and allowing a higher finesse. Consequently, the bow-tie cavity achieves the highest attenuation among the three candidates. Nevertheless, all three configurations satisfy the jitter attenuation requirement.

\begin{figure}
    \centering
    \includegraphics[width=1\linewidth]{Figs/3cav_jitter.pdf}
    \caption{Projected jitter noise for the three JAC design candidates. The PMC curve corresponds to an exact replica of the current aLIGO PMC, while the bow-tie and triangular cavity designs are newly developed configurations. All three satisfy the jitter attenuation requirement, suppressing the jitter noise to more than an order of magnitude below the A+ target sensitivity.}
    \label{fig:jitter_att}
\end{figure}

\subsubsection{Mispolarization effect}\label{subsubsec:mispol}
The resonance frequency splitting of the PMC caused by the birifringence has been reported to be \(\Delta \xi \approx 0.01\)~\cite{E2500324}. The estimated residual cavity-length fluctuation is $10^{-5}$--$10^{-4}$~nm in the observation band (Sec.~\ref{sec:length}), leading to RIN below $10^{-8}$ for a polarization misalignment of 5~deg. The corresponding estimation is shown in Fig.~\ref{fig:RINpol}.

Two main mechanisms are expected to contribute to residual mis-polarization. The first is the finite extinction ratio of the thin-film polarizer at the downstream end of the PSL. Assuming a 60~dB extinction ratio, the resulting polarization rotation is on the order of 1~mrad ($\sim 0.05$~deg). The second is imperfect alignment of the HAM1 input periscope, which rotates the beam polarization by 90~deg. A 1~deg misalignment of the periscope would introduce an additional $\sim 1$~deg of polarization error. Even under this conservative assumption, the resulting RIN remains below $10^{-8}$, and is therefore not a significant noise source relative to the A+ target sensitivity.

From these considerations, we conclude that the bow-tie cavity is not disadvantageous with respect to polarization-induced detuning.

\begin{figure}
    \centering
    \includegraphics[width=1\linewidth]{Figs/dualpol_RIN.pdf}
    \caption{Relative intensity noise due to residual cavity-length fluctuations under mixed polarization. The estimation uses the length noise from Sec.~\ref{sec:length} and assumes $\Delta\xi = 0.01$.}
    \label{fig:RINpol}
\end{figure}
\subsection{Conclusion}

Although the triangular cavity provides an intrinsic advantage with respect to polarization-induced detuning, the resulting intensity noise is expected to be negligible compared to the A+ target sensitivity. From a practical standpoint, adopting the same design as the current aLIGO PMC offers significant benefits: it eliminates the need for a new mechanical design, reduces lead time, and allows immediate use of the existing spare PMC units while additional spares are fabricated.

The jitter attenuation performance can be improved by replacing the PMC mirrors, but the gain is limited to approximately 10~dB. Even without this modification, the existing PMC design provides sufficient attenuation to reduce the jitter noise well below the requirement, achieving the A+ target sensitivity with a safety factor of 10. 

Based on these considerations, we recommend employing a JAC that replicates the current PMC design exactly.

\begin{table}
    \centering
    \begin{tabular}{c|cccc}
        & Triangular & Bow Tie & Bow Tie (PMC) & \\
        \hline\\
        Dimension & $4"\times0.6$ m & $4"\times0.5$ m & $4"\times0.5$ m &  \\
        Mirrors & 3 & 4 & 4 &\\
        Mirror curvature & 5.0 & 8.5 & 3.0 & m\\
        Input/output reflectivity & 2.25 & 0.98  & 2.48 & \%\\
        Finesse & 137 & 318 & 125 &\\
        Round-trip length & 1.31 & 2.02 & 2.02 & m\\
        FSR & 230 & 148 & 148 & MHz\\
        Waist size ($x/y$) & 755/756 & 799/802 & 546/549 & $\mu$m\\
        Rayleigh range ($x/y$) & 1.68/1.69 & 1.89/1.90 & 0.881/0.890 & m\\
        Divergence angle ($x/y$) & 449/448 & 424/422 & 620/617 & $\mu$rad\\
        Round-trip Gouy phase ($x/y$) & 222/42.2 & 57.0/56.8 & 101/100 & deg\\
        TEM10 (hor) attenuation & 1.22 & 1.03 & 1.63 &\%\\
        TEM01 (ver) attenuation & 3.16 & 0.98 & 1.64 &\%\\
    \end{tabular}
    \caption{Cavity parameters}
    \label{tab:JACdesign}
\end{table}
