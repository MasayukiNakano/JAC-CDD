\section{JAC control and sensing}
The Pound-Drever-Hall (PDH) method will be used for length control. The IO EOM located in the PSL will be used for the modulator, and the reflection from the JAC will be directed to the ISCT1/IOT1 and sense the signal with the LSC RFPD~\cite{E1900344_x0} located on the table. The digital servo will be used for the control. For the shot noise calculation, 2\% of mode-mismatching is assumed.

\subsection{Modulation}
The modulation frequency is set to 43 MHz, which is twice the modulation frequency used for the PSL FSS reference cavity (21.5 MHz). This frequency was chosen because the IO EOM has a 45 MHz channel, and the modulation index at 43 MHz is relatively large, approximately 0.1 rad/V \cite{G1900329_v3}. Additionally, generating the RF signal is straightforward using a frequency doubler~\cite{D1002178_v1}. The RF signal for modulation is taken directly from the distributor output, with a power level of 10 dBm, resulting in a modulation index of 0.01.

\subsection{Sensor}
The PD used to obtain the signal will be the LSC RFPD. The responsibility of the photodiode (Excelitas, C30642~\cite{ExcelitasC30642GHweb}) is approximately 0.82 A/W, with a quantum efficiency (QE) of 0.96 at 1064~nm, and a transimpedance is around 200 ohms. The RF beat signal will be demodulated with LSC I/Q demodulator \cite{E1200112_x0}, which has the conversion gain of 22 dB Vpp/Vpp \cite{T1300315_v1}. The injection power to the JAC is assumed as 100 W. The power relfected from the JAC and injected to the RFPD is adjusted with the HWP and PBS pair, ane we assumed the 20mW injected into RFPD with the 100W JAC injection. The PD noise is calculated with 2 mA of shotnoise intercept current\cite{G1101199_v1}.

The beat signal amplitude of the reflected light can be written as  
\begin{equation}
    P(\xi) = 4P_{DC}J_0(\beta)J_1(\beta) \frac{\xi-\xi_0}{1+(\xi-\xi_0)^2},
\end{equation}
where $\xi$ is the normalized frequency by the half-width-half-maximum (HWHM) $\xi_c$, $\xi_0$ is the resonant frequency, $J_n(\xi)$ is the Bessel function, and $\beta$ is the modulation index. The beat signal amplitude near resonance is approximately $4P_{DC}J_0(\beta)J_1(\beta)$. Thus, including the PD and I/Q demodulator parameters, the optical gain near the resonance can be calculated as 0.71 V/HWHM. The expected error signal calculated by Finesse is shown in Fig.~\ref{fig:JACPDH}.
\begin{figure}
    \centering
    \includegraphics[width=0.8\linewidth]{Figs/PDH_signal.pdf}
    \caption{PDH signal at the IQ demodulator calculated by Finesse model.}
    \label{fig:JACPDH}
\end{figure}
\subsection{Actuator}
The actuator will use a PZT attached to the JAC. We assume the use of a Noliac PAHH-0013 (PI) PZT, which has a DC response of 0.41 $\mu$m/200V. This value was calculated from the calibration measurement reported in~\cite{alog81385} and the DC gain 75 V/V of the PMC PZT driver. The high-voltage (HV) amplifier used to drive the PZT is \cite{E2500160_x0}, with a frequency response characterized by poles at 1 Hz and 500 Hz, a zero at 15 Hz, and a DC gain of 30dB. The transfer function is shown in \cite{T2500269_v1}

\subsection{Servo}
The control system will be implemented using a digital control system of CDS. Based on the data from T2000188, the phase delay caused by the CDS is expected to be approximately 300 µsec, which corresponds to a delay of about 5 cycles at a sampling rate of 16 kHz. Since the document assumes a direct connection between the ADC and DAC, so we should consider additional phase delays caused by model dependencies. So, we assume a time delay of 500 µsec, which corresponds to approximately 8 cycles. The achievable unity gain frequency (UGF) is determined by this phase delay. 

The servo will be designed with a phase margin requirement of 30 degrees. Assuming a first-order integrator and a first-order boost below 50 Hz, the open-loop gain will have a slope of $1/f^2$ below 50 Hz and $1/f$ above 50 Hz. Considering the phase response, an optimal UGF of around 200 Hz is expected. The Bode plot of the resulting open-loop transfer function is shown in Fig.~\ref{fig:OLG}. The phase margin is 40 degree.
\begin{figure}
    \centering
    \includegraphics[width=1\linewidth]{Figs/OLG.pdf}
    \caption{Open loop transfer function of the JAC LSC. Unity gain frequency is 200 Hz, and phase margin is 40 degrees. }
    \label{fig:OLG}
\end{figure}

\subsection{Length fluctuation}\label{sec:length}
The length fluctuation of the JAC was estimated based on the current length fluctuation of the PMC. Fig.~\ref{fig:PMClength} shows the PMC control signal multiplied by the actuator efficiency (3.7~µm/1000V). Since the PMC's length control bandwidth is in the kHz range, this serves as an estimate for the PMC's length fluctuation. However, as the JAC will be placed on the ISI and in a vacuum, this estimation overstates the length fluctuation for the JAC, particularly in the audio frequency band where acoustic noise typically dominates. Additionally, above 300~Hz, the spectrum appears to be limited by sensing or actuator noise. Therefore, we modeled the JAC's expected length fluctuation as shown by the green solid line in Fig.~\ref{fig:PMClength}.


Fig.~\ref{fig:errorsig} shows the spectrum of the signal obtained from the JAC reflection sensor, including length fluctuation, shot noise, PD noise \cite{??}, and ADC noise. The ADC noise was calculated assuming two-stage whitening. The shot noise is calculated by assuming 2\% of mode-mismatching, which gives about 0.9 mW DC power at the REFL PD.  As shown in the figure, the signal strength is sufficient. The noise budget of the length fluctuation suppressed by the designed servo is shown in Fig.~\ref{fig:noise_budget}.

\begin{figure}
    \centering
    \includegraphics[width=1\linewidth]{Figs/PMC_length.pdf}
    \caption{PMC length fluctuation of each site. This plot is using the data of L/H1:PSL-PMC\_HV\_MON\_OUT\_DQ from GPS time 1411790418 for 300 seconds. From these curves, the modeled JAC length motion is shown as solid green line.}
    \label{fig:PMClength}
\end{figure}

\begin{figure}
    \centering
    \includegraphics[width=1\linewidth]{Figs/ErrorSpe.pdf}
    \caption{Expected error signal spectrum. For the shot noise calculation, 2\% of mode-mismatching is assumed. The PD noise is calculated with 2 mA of shotnoise intercept current\cite{}. Demodulator and ADC noise refer \cite{}, and two-stages of the whitening filter is assumed. }
    \label{fig:errorsig}
\end{figure}

\begin{figure}
    \centering
    \includegraphics[width=1\linewidth]{Figs/NoiseBudget.pdf}
    \caption{Noise budget of the residual length motion.}
    \label{fig:noise_budget}
\end{figure}
\subsection{Intensity noise due to residual length fluctuation}
To evaluate whether the suppression provided by the designed control loop is sufficient, we assess the intensity noise caused by residual length fluctuations. Same as discussed in Sec.~\ref{sec:pol}, detuned cavities convert length fluctuations into intensity noise through the transfer function given similar to Eq.~\eqref{eq:11_20_18_18_16} as
\begin{equation}
\begin{aligned}\label{eq:11_25_19_00_21}
    \frac{dP}{d\xi} = 2\Delta\xi,
\end{aligned}
\end{equation}
where $\Delta\xi$ is detunning normalized with HWHM. The HWHM in the unit of length is 2.1 nm. The previous calculation showed that length fluctuations are suppressed to below \(3\times10^{-14}\) m across all frequency ranges. Normalized to HWHM, this fluctuation corresponds to \(1.5 \times 10^{-5}\). The intensity noise requirement for the JAC is to remain below the intensity noise stabilized by the current ISS inner loop, which is approximately \(5 \times 10^{-8}\) at 100 Hz as the relative intensity noise. From this, we derive that if the detuning of the cavity is less than \(1.6 \times 10^{-3}\times\)HWHM, or equivalently, an offset of \(1.1\) mV in the demodulator output, then the output intensity noise would be comparable to the current residual intensity noise of the ISS inner loop.

Fig.~\ref{fig:RIN_length} shows the expected intensity noise caused by length fluctuation with different offsets, along with the inner-loop-stabilized intensity noise measured at Hanford. As shown in the figure, intensity noise remains at acceptable levels if the offset drift stays below 2 mV.

\begin{figure}
    \centering
    \includegraphics[width=1\linewidth]{Figs/RIN_length.pdf}
    \caption{Expected relative intensity noise (RIN) caused by the residual JAC length motion. The current intensity noise stablized by the ISS inner-loop is shown as red solid line~\cite{T1600064_v2}. }
    \label{fig:RIN_length}
\end{figure}

\subsection{Phase noise due to residual length fluctuation}
The residual length noise will introduce phase noise into the output beam as well. The amplitude transmissivity of the critically coupled cavity is
\begin{equation}
    t_\text{cav}(\xi) = \frac{1}{1-i\xi} = \frac{1+i\xi}{1+\xi^2},
\end{equation}
where $\xi$ is the residual length noise normalized by Half-Width-Half-Maximum (HWHM). Therefore, the phase noise of the output beam can be expressed as
\begin{equation}
    \Delta\phi_\text{out} = \arctan\xi \simeq \xi.
\end{equation}
A residual length fluctuation of $10^{-13}$~m will be converted into $4.7\times10^{-5}$ rad of phase noise, which is equivalent to $4.7\times10^{-3}$~Hz of frequency noise at 1~kHz.

Fig.~\ref{fig:freqNoise} shows the estimated frequency noise caused by the residual length fluctuation of the JAC. The frequency noise will be suppressed by the CARM loop, and the suppressed frequency noise due to the JAC motion is also illustrated. This shows that the resulting frequency noise will be comparable to the current CARM residual noise.

\begin{figure}
    \centering
    \includegraphics[width=1\linewidth]{Figs/FreqNoise_length.pdf}
    \caption{Frequency noise caused by the residual length noise of the JAC. The black dashed line shows the aLIGO FSS requirement~\cite{T0900649_v4}. Orange and green lines represent the frequency noise of the JAC output with and without the CARM loop, respectively. The CARM loop will reduce the output frequency noise to a level comparable to the current frequency noise of the input beam into the PRM.}
    \label{fig:freqNoise}
\end{figure}

\subsection{Slow control}
The JAC has two heaters of IS250C50RKE OHMITE. The slow drift of the cavity length (less than 0.1 Hz) will be offloaded to the JAC heater. During the conceptual design process, heat conduction to the table was considered as a potential issue, but we believe this will not pose a problem. The heater will maintain the JAC temperature, compensating for heating by the full laser power during lock acquisition. The expected use of the heater is as follows:
\begin{enumerate}
    \item Lock the JAC with PZT at low power (about 1 W), and offload the DC control to the heater to keep the PZT input constant.
    \item During lock acquisition, maintain this slow loop. As the heat from the laser increases with more input power, this loop will keep the JAC length stable by reducing the heater output.
\end{enumerate}
As a result, the total heat on the JAC will remain constant during the lock acquisition. This control scheme allows the JAC heater to prevent changes in heat conduction to the ISI table.

 
\subsection{Conclusion of the length control}
For the length control, our design will include the following:
\begin{itemize}
    \item Digital control.
    \item Bandwidth of 200 Hz.
    \item Required offset drift at I/Q demodulator less than 2 mV.
    \item 50 mW of power will be injected into the RF PD.
    \item The resulting equivalent intensity and frequency noise will be comparable to the current PSL output beam.
\end{itemize}

\subsection{Alignment control}

TBD
