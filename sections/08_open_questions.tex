\section{Open questions}
\subsection{Thermal effect of the JAC}
While the thermal lensing of the EOM has been verified to not pose a significant issue, the thermal effects of the JAC itself remain untested. As the input power changes from a few watts to over a hundred watts, the thermal conditions within the JAC will shift drastically. Although the heater will be used as part of the thermal compensation system, local thermal gradients, such as thermal lensing of the input or output mirrors and deformation of the JAC body, cannot be fully avoided. These effects may lead to changes in the output mode or alignment drift. Alignment compensation may be achievable through the tip-tilt suspensions, but if the available compensation range proves insufficient, it may be necessary to design an in-vacuum power adjustment stage. 

\subsection{Length fluctuation}
The design calculations are based on assumed length fluctuations, estimated from the in-air PMC length fluctuations. However, it is possible that these assumptions underestimate the actual length fluctuations, leading to higher intensity and frequency noises in the output beam. During the testing phase, it will be necessary to evaluate the actual length fluctuation. If it is found to be larger than expected, we may need to increase the control bandwidth beyond the current design, potentially up to the kHz range. This could be achieved by switching from digital control to an analog servo system.

Related to the length fluctuation, the PMC is equipped with a damping structure designed to suppress body resonance. This structure utilizes Viton sheets, clamped from above, to apply force and mitigate resonant modes. It may require redesigning to improve performance and ensure compatibility with vacuum conditions. 
